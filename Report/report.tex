%%%%%%%%%%%%%%%%%%%%%%%%%%%%%%%%%%%%%%%%%
% University/School Laboratory Report
% LaTeX Template
% Version 3.1 (25/3/14)
%
% This template has been downloaded from:
% http://www.LaTeXTemplates.com
%
% Original author:
% Linux and Unix Users Group at Virginia Tech Wiki 
% (https://vtluug.org/wiki/Example_LaTeX_chem_lab_report)
%
% License:
% CC BY-NC-SA 3.0 (http://creativecommons.org/licenses/by-nc-sa/3.0/)
%
%%%%%%%%%%%%%%%%%%%%%%%%%%%%%%%%%%%%%%%%%

%----------------------------------------------------------------------------------------
%	PACKAGES AND DOCUMENT CONFIGURATIONS
%----------------------------------------------------------------------------------------

\documentclass{article}

%\usepackage[version=3]{mhchem} % Package for chemical equation typesetting
%\usepackage{siunitx} % Provides the \SI{}{} and \si{} command for typesetting SI units
\usepackage{graphicx} % Required for the inclusion of images
%\usepackage{natbib} % Required to change bibliography style to APA
%\usepackage{amsmath} % Required for some math elements 
\usepackage{listings}
\usepackage{tikz}

\setlength\parindent{0pt} % Removes all indentation from paragraphs

\renewcommand{\labelenumi}{\alph{enumi}.} % Make numbering in the enumerate environment by letter rather than number (e.g. section 6)

%\usepackage{times} % Uncomment to use the Times New Roman font

%----------------------------------------------------------------------------------------
%	DOCUMENT INFORMATION
%----------------------------------------------------------------------------------------

\title{Report: Homework 1 - Parallel Movie Rendering with Pov-Ray}% Title

\author{Jan \textsc{Schlenker}} % Author name

\date{\today} % Date for the report

\begin{document}

\maketitle % Insert the title, author and date

\begin{center}
\begin{tabular}{l l}
Instructor: & Dipl.-Ing. Dr. Simon Ostermann \\
Parts solved of the sheet: & Tasks 1-8 \\
\end{tabular}
\end{center}

% If you wish to include an abstract, uncomment the lines below
% \begin{abstract}
% Abstract text
% \end{abstract}

%----------------------------------------------------------------------------------------
%	SECTION 1
%----------------------------------------------------------------------------------------

\section{How to run the programme}

First of all extract the archive file homework\_1.tar.gz:

\begin{lstlisting}[language=bash, deletekeywords={cd}]
  $ tar -xzf homework_1.tar.gz
  $ cd homework_1
\end{lstlisting}

Afterwards move/copy the binary files gm and povray to the bin/ directory and the files scherk.args, scherk.ini and scherk.pov to the inputdata/ directory:

\begin{lstlisting}[language=bash]
  $ cp <gm-file-path> <povray-file-path> bin/
  $ cp <scherk-files-dir>/scherk* inputdata/
\end{lstlisting}

Now you can run the main programme:

\begin{lstlisting}[language=bash]
  $ ./main.sh
\end{lstlisting}

%\begin{center}\ce{2 Mg + O2 -> 2 MgO}\end{center}

% If you have more than one objective, uncomment the below:
%\begin{description}
%\item[First Objective] \hfill \\
%Objective 1 text
%\item[Second Objective] \hfill \\
%Objective 2 text
%\end{description}

%----------------------------------------------------------------------------------------
%	SECTION 2
%----------------------------------------------------------------------------------------

\section{Programme explanation}
The files of the the programme are structured as follows:

\begin{itemize}
\item The \textbf{main.sh} script contains the main programme
\item The \textbf{bin} directory contains the binaries povray and gm
\item The \textbf{inputdata} directory contains the necessary files for the povray binary
\item The \textbf{jobs} directory contains the jobs which are executed by the main script via qsub
\end{itemize}

The main programme main.sh and the jobs are written in shell script. One advantage of shell script for the tasks is that grid engine commands like qsub and qhost are directly available in shell script.
\\
\\
Below is the programme explanation task by task:

\begin{itemize}
\item \textbf{Task 1:} To render all png-files with one processor in the grid the programme uses the qsub command \underline{once} to execute the povray job under jobs/job\_task1.pbs. The povray job itself uses the INI-file under inputdata/scherk.ini to determine how many png-files should be rendered. For task 2 it is important to give the job a name.
\item \textbf{Task 2:} The programme waits until all png-files are rendered. To achieve this, image merge script (bin/gm) is submitted as a job as well, which waits with the help of the -hold-jid flag for job\_task1. The main programme itself waits with the help of the -sync flag. When job\_task1 finishes, job\_task2 will start to merge all images.
\item \textbf{Task 3:} For the measurement the programme subtracts a timestamp after task 2 from a timestamp before task 1.
\item \textbf{Task 4 \& 5:} First of all the programme extracts the number of frames (M) which should be rendered from the INI-file. Then it extracts the number of available processors (N) in the grid via the qhost command. If there are more processors than frames, only M processors will be used for rendering. Otherwise the frames are equally split on the processors following the rule from the task sheet. To achieve this, the qsub command is used \underline{multiple times} to execute the povray job with the parameters Subset\_Start\_Frame and Subset\_End\_Frame.
\item \textbf{Task 6:} For the measurement the programme subtracts a timestamp after task 5 from a timestamp before task 4.
\item \textbf{Task 7 \& 8:} The results are executed and presented with the help of the bc and awk command.
\end{itemize}


%----------------------------------------------------------------------------------------
%	SECTION 3
%----------------------------------------------------------------------------------------

\section{Results}

The test environment consisted of a grid of 40 processors in total. Measurements were made for 2, 4, 8, 16, 32, 64 and 128 frames. Table~\ref{tab:measurements} on page~\pageref{tab:measurements} shows the sequential and parallel execution times as well as the speedup and the efficency for the frame numbers.
\\
\\
While T$_{seq}$ is roughly doubled with every doubling of the frames, T$_{par}$ is increased much slower. Either way the increase of T$_{par}$ gets bigger with the increase of the frames. This is due to the sequential merge in T$_{par}$ (cf. Amdahl's law). The speedup and the efficency are increased quite steadily until the frame number is bigger than the number of processors. Then more and more processors have to render more than one frame, so that the total execution time gets bigger.
\\
\\
All in all the results show that the use of multiple processors can decrease the rendering time of the pictures enormously and that the time difference of T$_{seq}$ and T$_{par}$ gets bigger with the number of frames.  

\begin{table}[htbp]
\centering
\begin{tabular}{ | c | p{1cm} | c | c | c | c |}
\hline
\textbf{Frames} & \textbf{CPUs for T$_{par}$} & \textbf{T$_{seq}$ in sec} & \textbf{T$_{par}$ in sec} & \textbf{Speedup} & \textbf{Efficency} \\
\hline \hline
2 & 2 & 38,15 & 23,86 & 1,60 & 0,04 \\
\hline
4 & 4 & 76,18 & 27,89 & 2,73 & 0,07 \\
\hline
8 & 8 & 147,47 & 35,90 & 4,11 & 0,10 \\
\hline
16 & 16 & 292,41 & 49,91 & 5,85 & 0,14 \\
\hline
32 & 32 & 582,99 & 75,93 & 7,68 & 0,19 \\
\hline
64 & 40 & 1161,30 & 141,94 & 8,18 & 0,20 \\
\hline
128 & 40 & 2319,09 & 277,01 & 8,37 & 0,21 \\
\hline
\end{tabular}
\caption{Measurements}
\label{tab:measurements}
\end{table}

\vfill
\textbf{Total points:} \hspace{1em} /10

%Because of this reaction, the required ratio is the atomic weight of magnesium: \SI{16.00}{\gram} of oxygen as experimental mass of Mg: experimental mass of oxygen or $\frac{x}{1.31}=\frac{16}{0.87}$ from which, $M_{\ce{Mg}} = 16.00 \times \frac{1.31}{0.87} = 24.1 = \SI{24}{\gram\per\mole}$ (to two significant figures).

%----------------------------------------------------------------------------------------
%	SECTION 4
%----------------------------------------------------------------------------------------

%\section{Results and Conclusions}

%The atomic weight of magnesium is concluded to be \SI{24}{\gram\per\mol}, as determined by the stoichiometry of its chemical combination with oxygen. This result is in agreement with the accepted value.

%\begin{figure}[h]
%\begin{center}
%\includegraphics[width=0.65\textwidth]{placeholder} % Include the image placeholder.png
%\caption{Figure caption.}
%\end{center}
%\end{figure}
%
%----------------------------------------------------------------------------------------
%	SECTION 5
%----------------------------------------------------------------------------------------

%\section{Discussion of Experimental Uncertainty}

%The accepted value (periodic table) is \SI{24.3}{\gram\per\mole} \cite{Smith:2012qr}. The percentage discrepancy between the accepted value and the result obtained here is 1.3\%. Because only a single measurement was made, it is not possible to calculate an estimated standard deviation.

%The most obvious source of experimental uncertainty is the limited precision of the balance. Other potential sources of experimental uncertainty are: the reaction might not be complete; if not enough time was allowed for total oxidation, less than complete oxidation of the magnesium might have, in part, reacted with nitrogen in the air (incorrect reaction); the magnesium oxide might have absorbed water from the air, and thus weigh ``too much." Because the result obtained is close to the accepted value it is possible that some of these experimental uncertainties have fortuitously cancelled one another.


%----------------------------------------------------------------------------------------
%	BIBLIOGRAPHY
%----------------------------------------------------------------------------------------

%\bibliographystyle{apalike}

%\bibliography{sample}

%----------------------------------------------------------------------------------------


\end{document}
